\documentclass{article}
\usepackage[utf8]{inputenc}
\usepackage[left=1in,right=1in,top=1in,bottom=1in]{geometry}
\usepackage{crop,graphicx,array,color,flushend,stfloats,amsthm,chngpage,times,fancyhdr,lipsum,lastpage}

%%%%%%%%%%%%   Extra libraries & settings %%%%%%%%%%%%%
\setlength{\parskip}{0.25em}

%%%%%%%%%%%%   Header and Footer  %%%%%%%%%%%%%
\pagestyle{fancy}

\fancypagestyle{plain}{%
  \renewcommand{\headrulewidth}{0pt}%
  \fancyhf{}%
  \fancyfoot[R]{Page \bf\thepage\ \rm of \bf\pageref{LastPage}}%
}


%%%% Customise Titles and Headers: %%%%
\title{Don't forget adding title}
\author{Chinchuthakun Worameth (18B00033)}
\date{\today}

\fancyhf{}
\fancyhead[L]{Chinchuthakun Worameth}
\fancyhead[R]{18B00033}
\fancyfoot[R]{Page \bf\thepage\ \rm of \bf\pageref{LastPage}}


\AtBeginDocument{
%%%%%%%%%%%% Make Title and Format Lines %%%%%%%%%%%%
\maketitle											%
\vspace{-120px}										%
\noindent\rule{\linewidth}{1pt} \par				%
\vspace{100px}										%
\vspace{-20px}										%
\noindent\rule{\linewidth}{1pt} \par				%
\vspace{10px}										%
% %%%%%%%%%%%%%%%%%%% Content %%%%%%%%%%%%%%%%%%%%	
}
\input{math-command}
\title{Quiz \#2: Supplementary Document}
\begin{document}
\section{Question \#1}
Let $\val[G]=(\val[V],\val[E])$ and $\val[A]$ be an undirected graph and an adjacency matrix associated with $\val[G]$, respectively. Since there exists a walk\footnote{Here, I refer to \tit{walk} instead of \tit{path} because nodes and edges may be repeated in this manner of calculation.} of length $k$ between node $i$ and $j$ if and only if $\val[A][][ij]^k = 1$, one might be tempted to conclude that the number of triangles, i.e. cycle of length $3$, in the graph is $\Tr\parens*{\val[A]^k}$. However, it is, in fact, false because each cycle is counted 6 times instead of only once. Mathematically, let $e_1, e_2, e_3 \in \val[E]$ be edges in a triangle, every permutation of $\{e_1,e_2,e_3\}$ is equivalent. Therefore, we need to account for it by multiplying $1/6$:
\begin{equation}
    \text{\#triangle} = \frac{1}{6} \Tr\parens*{\val[A]^k}
\end{equation}

Incidentally, this formula cannot be generalized for cycles of length $n>3$ because a cycle is defined as a trail, i.e. walk without repeated edges, in which only the first and last nodes are the same. It is only true for triangles because all closed walks of length $3$ must be a trail. That is, given that $u,A,B,u$ as a sequence of visited nodes, $A$ and $B$ can neither be node $u$ nor the same node because $\val[G]$ contains no self-loop. Therefore, all nodes are unique, implying that all edges are unique since $\val[G]$ is not a multigraph.

\section{Question \#2}
Suppose that $\val[G]=(\val[V],\val[E])$ is an undirected graph without self-loops and multiedges, we can verify that nodes $u,v,w \in \val[V]$ form a triangle if it is a 3-clique. Therefore, $\val[G]$ contains the maximum possible number of triangle if all combinations of $u,v,w \in \val[V]$ form $K_3$
\begin{equation}
    \text{\#max\_triangle} = \genfrac(){0pt}{0}{\abs{\val[V]}}{3}
\end{equation}

\section{Question \#3}
According to Turan's theorem, a graph $\val[G]=(\val[V],\val[E])$ without a $(k+1)$-clique must satisfy the inequality
\begin{equation}
    \abs{E} \leq \frac{(k-1)\abs{V}^2}{2k}
\end{equation}
For $k=2$, graph with $\floor*{\frac{(k-1)\abs{V}^2}{2k}}$ edges is a complete bipartite graph such that the two sets of nodes have as equal size as possible. Therefore, we generate a graph of $n = 9$ nodes and 12 edges that contains no triangles by:
\begin{enumerate}
    \item Divide nodes into 2 groups, $S_1 = \cbracket{1,\ldots,\floor{n/2}}$ and $S_2 = \cbracket{\ceil{n/2},\ldots,n}$.
    \item Add an edge between a pair of nodes $(u,v)$ where $u \in S_1$ and $v \in S_2$ until we cannot do it anymore or $\abs{E} = m$.
\end{enumerate}

\end{document}