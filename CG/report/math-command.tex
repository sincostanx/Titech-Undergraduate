% math stuff
\usepackage{amsmath}                % to use DeclareMathOperator
\usepackage{amssymb}
\usepackage{mathrsfs}
\usepackage{mathtools}
\usepackage{xargs}                  % for more than one optional arguments when define new commands
\usepackage{physics}                % vectors
\usepackage{mdframed}               % frames for definition, theorem, etc.
\usepackage[ruled]{algorithm2e}     % for algorithms
\usepackage{ifthen}                % for \ifthenelse
\usepackage{upgreek}

% figures and tables
\usepackage{titlesec}
\usepackage{caption}
\usepackage{subcaption}
\usepackage{multirow}
\usepackage{relsize}

% formatting some important single letters
\renewcommand{\epsilon}{\varepsilon}
\renewcommand{\phi}{\varphi}
\renewcommand{\upepsilon}{\upvarepsilon}
\renewcommand{\upphi}{\upvarphi}

% new operators
\DeclareMathOperator*{\argmin}{arg\,min}                % argmin
\DeclareMathOperator*{\argmax}{arg\,max}                % argmax
\DeclarePairedDelimiter\ceil{\lceil}{\rceil}            % ceiling function
\DeclarePairedDelimiter\floor{\lfloor}{\rfloor}         % floor function
\DeclarePairedDelimiter{\parens}{\lparen}{\rparen}      % parenthesis (use \parens* for automatically adjusting version)
\DeclarePairedDelimiter{\bracket}{[}{]}
\DeclarePairedDelimiter{\cbracket}{\{}{\}}
\DeclarePairedDelimiter{\ang}{\langle}{\rangle}
% \DeclarePairedDelimiter{\fourier}{\mathscr{F}\{}{\}}
% \DeclarePairedDelimiter{\invfourier}{\mathscr{F}^{-1}\{}{\}}

\newcommand{\fourier}[1]{\mathscr{F}\cbracket*{#1}}
\newcommand{\invfourier}[1]{\mathscr{F}^{-1}\cbracket*{#1}}
\newcommand {\dx}{\,dx}
\newcommand {\dy}{\,dy}
\newcommand {\dz}{\,dz}
\newcommand {\dt}{\,dt}
\newcommand {\du}{\,du}
\newcommand {\dtheta}{\,d\theta}
\newcommand {\domega}{\,d\omega}



\newcommand{\bigo}[1]{\ensuremath{\mathcal{O}\parens*{#1}}}
% new commands
\newcommand{\st}{such that }
\newcommand{\w}{where }
\newcommand{\del}{\nabla}
\newcommand{\larrow}{\leftarrow}
\newcommand{\rarrow}{\rightarrow}
\newcommand{\tbf}{\textbf}
\newcommand{\tit}{\textit}
\newcommand{\col}{\operatorname{col}}
\newcommand{\mat}[1]{\begin{matrix} #1 \end{matrix}}
% \newcommand{\vect}[1]{\boldsymbol{\mathbf{#1}}}
\newcommand{\vf}[1]{\boldsymbol{\mathbf{#1}}}
\newcommandx*{\seq}[2][1,2]{\ensuremath{#1, \ldots, #2}}
\newcommandx*{\ssum}[3][1,2,3]{\ensuremath{\sum_{#1 = #2}^{#3}}}
\newcommandx*{\sint}[2][1,2]{\ensuremath{\int_{#1}^{#2}}}
% \newcommandx*{\func}[4][1,2,3,4]{\ensuremath{#1^{\parens{#2}}_{#3}\parens{#4}}}
% \newcommandx*{\val}[3][1,2,3]{\ensuremath{#1^{\parens{#2}}_{#3}}}

% \newcommandx*{\func}[4][1=f,2=x,3,4, usedefault]{
%     \ifthenelse{\equal{#3}{}}{\ensuremath{#1_{#4}\parens{\vf{#2}}}}{\ensuremath{#1^{\parens{#3}}_{#4}\parens{\vf{#2}}}}
% }
\newcommandx*{\func}[3][1=f,2,3, usedefault]{
    \ifthenelse{\equal{#2}{}}{\ensuremath{#1_{#3}}}{\ensuremath{#1^{\parens{#2}}_{#3}}}
}
\newcommandx*{\val}[3][1,2,3, usedefault]{
    \ifthenelse{\equal{#2}{}}{\ensuremath{\vf{#1}_{#3}}}{\ensuremath{\vf{#1}^{\parens{#2}}_{#3}}}
}

\newcommandx*{\Real}[1][1, usedefault]{\ensuremath{\mathbb{R}^{#1}}}                % set of real number
\newcommandx*{\Int}[1][1, usedefault]{\ensuremath{\mathbb{Z}^{#1}}}                 % set of integer
\newcommandx*{\Natural}[1][1, usedefault]{\ensuremath{\mathbb{N}^{#1}}}             % set of natural number
\newcommandx*{\normal}[2][1=0, 2=1, usedefault=!]{\ensuremath{\mathcal{N}(#1,#2)}}  % Gaussian distribution

% define frame environment
% \newmdtheoremenv{definition}{Definition}
% \newmdtheoremenv{proposition}{Proposition}
% \newmdtheoremenv{corollary}{Corollary}
% \newmdtheoremenv{lemma}{Lemma}
% \newmdtheoremenv{theorem}{Theorem}
% \newmdtheoremenv{remark}{Remark}

% define keywords for algorithm
\SetKwInOut{Input}{Input}
\SetKwInOut{Output}{Output}
\SetKwInOut{Parameter}{Parameter}

% \begin{theorem}{text}{label}
% refer as \ref{tha:label}
\usepackage{tcolorbox}
\tcbuselibrary{theorems,breakable} %% を読み込む
\definecolor{burgundy}{rgb}{0.5, 0.0, 0.13}
\newtcbtheorem[number within=section]{theorem}{Theorem}%
{colframe=burgundy,colback=burgundy!2!white,
rightrule=0pt,leftrule=0pt,bottomrule=2pt,
colbacktitle=burgundy,theorem style=standard,breakable,arc=0pt}{theo}

\definecolor{oxfordblue}{rgb}{0.0, 0.13, 0.28}
\newtcbtheorem[number within=section]{definition}{Definition}%
{colframe=oxfordblue,colback=oxfordblue!2!white,
rightrule=0pt,leftrule=0pt,bottomrule=2pt,
colbacktitle=oxfordblue,theorem style=standard,breakable,arc=0pt}{def}

\definecolor{cadmiumorange}{rgb}{0.93, 0.53, 0.18}
\newtcbtheorem[number within=section]{remark}{Remark}%
{colframe=cadmiumorange,colback=cadmiumorange!2!white,
rightrule=0pt,leftrule=0pt,bottomrule=2pt,
colbacktitle=cadmiumorange,theorem style=standard,breakable,arc=0pt}{rem}

% equation numbering
\numberwithin{equation}{section}

%%%%%%%%%%%%%%%%%%%%%%%%%%%
\usepackage{listings}
\usepackage{xcolor}

\definecolor{codegreen}{rgb}{0,0.6,0}
\definecolor{codegray}{rgb}{0.5,0.5,0.5}
\definecolor{codepurple}{rgb}{0.58,0,0.82}
\definecolor{backcolour}{rgb}{0.95,0.95,0.92}

\lstdefinestyle{mystyle}{
    backgroundcolor=\color{backcolour},   
    commentstyle=\color{codegreen},
    keywordstyle=\color{magenta},
    numberstyle=\tiny\color{codegray},
    stringstyle=\color{codepurple},
    basicstyle=\ttfamily\footnotesize,
    breakatwhitespace=false,         
    breaklines=true,                 
    captionpos=b,                    
    keepspaces=true,                 
    numbers=left,                    
    numbersep=5pt,                  
    showspaces=false,                
    showstringspaces=false,
    showtabs=false,                  
    tabsize=2
}

\lstset{style=mystyle}

\makeatletter
\renewcommand*\env@matrix[1][\arraystretch]{%
  \edef\arraystretch{#1}%
  \hskip -\arraycolsep
  \let\@ifnextchar\new@ifnextchar
  \array{*\c@MaxMatrixCols c}}
\makeatother