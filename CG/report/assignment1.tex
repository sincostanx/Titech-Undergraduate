\documentclass{article}
\usepackage[utf8]{inputenc}
\usepackage[left=1in,right=1in,top=1in,bottom=1in]{geometry}
\usepackage{crop,graphicx,array,color,flushend,stfloats,amsthm,chngpage,times,fancyhdr,lipsum,lastpage}

%%%%%%%%%%%%   Extra libraries & settings %%%%%%%%%%%%%
\setlength{\parskip}{0.25em}

%%%%%%%%%%%%   Header and Footer  %%%%%%%%%%%%%
\pagestyle{fancy}

\fancypagestyle{plain}{%
  \renewcommand{\headrulewidth}{0pt}%
  \fancyhf{}%
  \fancyfoot[R]{Page \bf\thepage\ \rm of \bf\pageref{LastPage}}%
}


%%%% Customise Titles and Headers: %%%%
\title{Don't forget adding title}
\author{Chinchuthakun Worameth (18B00033)}
\date{\today}

\fancyhf{}
\fancyhead[L]{Chinchuthakun Worameth}
\fancyhead[R]{18B00033}
\fancyfoot[R]{Page \bf\thepage\ \rm of \bf\pageref{LastPage}}


\AtBeginDocument{
%%%%%%%%%%%% Make Title and Format Lines %%%%%%%%%%%%
\maketitle											%
\vspace{-120px}										%
\noindent\rule{\linewidth}{1pt} \par				%
\vspace{100px}										%
\vspace{-20px}										%
\noindent\rule{\linewidth}{1pt} \par				%
\vspace{10px}										%
% %%%%%%%%%%%%%%%%%%% Content %%%%%%%%%%%%%%%%%%%%	
}
\input{math-command}
\title{Computer Graphics Assignment \#1}
\begin{document}

\section{Question}
Answer the inverse Fourier transform of the functions:
$$
\begin{gathered}
    \func[F][][1](\omega_x,\omega_y) = \func[\delta](\omega_x - \sqrt{3})\func[\delta](\omega_y - 1) + \func[\delta](\omega_x + \sqrt{3})\func[\delta](\omega_y + 1) \\
    \func[F][][2](\omega_x,\omega_y) = i\func[\delta](\omega_x - \sqrt{3})\func[\delta](\omega_y - 1) - i\func[\delta](\omega_x + \sqrt{3})\func[\delta](\omega_y + 1)
\end{gathered}
$$

\section{Answer}
\subsection{Equation}
\subsubsection{Inverse Fourier transform of $\func[F][][1]$}

\begin{align}
\invfourier{\func[F][][1](\omega_x,\omega_y)} 
&= \invfourier{\func[\delta](\omega_x - \sqrt{3})\func[\delta](\omega_y - 1) + \func[\delta](\omega_x + \sqrt{3})\func[\delta](\omega_y + 1)} \\
&= \invfourier{\func[\delta](\omega_x - \sqrt{3})\func[\delta](\omega_y - 1)} + \invfourier{\func[\delta](\omega_x + \sqrt{3})\func[\delta](\omega_y + 1)}
\end{align}
By applying the property of the dirac delta function $\sint[-\infty][\infty] \func[f](x)\func[\delta](x - x_0) \dx = \func[f](x_0)$, we have
\begin{align}
\invfourier{\func[\delta](\omega_x - \sqrt{3})\func[\delta](\omega_y - 1)}
&= \frac{1}{4\pi^2} \sint[-\infty][\infty]  \func[\delta](\omega_y - 1) e^{i\omega_y y} \sint[-\infty][\infty] \func[\delta](\omega_x - \sqrt{3}) e^{i\omega_x x} \domega_x  \domega_y \\
&= \frac{1}{4\pi^2} \sint[-\infty][\infty]  \func[\delta](\omega_y - 1) e^{i\omega_y y} \parens*{e^{\sqrt{3} ix}}  \domega_y \\
&= \frac{1}{4\pi^2} \parens*{e^{\sqrt{3} ix}} \parens*{e^{iy}} \label{first-term}
\end{align}
and similarly,
\begin{equation}
    \invfourier{\func[\delta](\omega_x + \sqrt{3})\func[\delta](\omega_y + 1)} = \frac{1}{4\pi^2} \parens*{e^{-\sqrt{3} ix}} \parens*{e^{-iy}} \label{second-term}
\end{equation}
Therefore, by combining result from \eqref{first-term} and \eqref{second-term} and applying the Euler's formula, we have
\begin{align}
\invfourier{\func[F][][1](\omega_x,\omega_y)} 
&= \frac{1}{4\pi^2} \parens*{{e^{\sqrt{3} ix + iy}} + {e^{-\sqrt{3} ix - iy}}} \label{combine} \\
&= \frac{1}{4\pi^2} \parens*{\cos(\sqrt{3}x + y) + i\sin(\sqrt{3}x + y) + \cos(-\sqrt{3}x - y) + i\sin(-\sqrt{3}x - y)} \\
&= \frac{1}{4\pi^2} \parens*{\cos(\sqrt{3}x + y) + i\sin(\sqrt{3}x + y) + \cos(\sqrt{3}x + y) - i\sin(\sqrt{3}x + y)} \\
&= \frac{1}{2\pi^2} \cos(\sqrt{3}x + y) \label{result-1}
\end{align}

\subsubsection{Inverse Fourier transform of $\func[F][][2]$}

Since $\func[F][][1]$ and $\func[F][][2]$ are the same, except the coefficient $i$ and the minus sign in the latter, we can modify the expression in \eqref{combine}:
\begin{align}
\invfourier{\func[F][][2](\omega_x,\omega_y)} 
&= \frac{i}{4\pi^2} \parens*{{e^{\sqrt{3} ix + iy}} - {e^{-\sqrt{3} ix - iy}}} \\
&= \frac{i}{4\pi^2} \parens*{\cos(\sqrt{3}x + y) + i\sin(\sqrt{3}x + y) - \cos(-\sqrt{3}x - y) - i\sin(-\sqrt{3}x - y)} \\
&= \frac{i}{4\pi^2} \parens*{\cos(\sqrt{3}x + y) + i\sin(\sqrt{3}x + y) - \cos(\sqrt{3}x + y) + i\sin(\sqrt{3}x + y)} \\
&= -\frac{1}{2\pi^2} \sin(\sqrt{3}x + y) \label{result-2}
\end{align}

\subsection{Graph}
We can visualize the inverse Fourier transform of $\func[F][][1]$ and $\func[F][][2]$, described in \eqref{result-1} and \eqref{result-2}, by using Python.

\begin{figure}[h]
    \centering
    \includegraphics[width=0.9\textwidth]{figures/assignment1/cg1-assignment1.png}
    \caption{Visualization of $\invfourier{\func[F][][1](\omega_x,\omega_y)}$ and $\invfourier{\func[F][][2](\omega_x,\omega_y)}$}
\end{figure}

\subsection{Text (Full sentences)}
\begin{itemize}
    \item $\invfourier{\func[F][][1](\omega_x,\omega_y)}$ is a cosine function with an amplitude of $\frac{1}{2\pi}$ in the direction of 30 degrees from $x$-axis. Its period is $\parens*{\frac{2\pi}{\sqrt{3}}, 2\pi}$.
    \item Similarly, $\invfourier{\func[F][][2](\omega_x,\omega_y)}$ is a sine function with an amplitude of $\frac{1}{2\pi}$ in the direction of 30 degrees from $x$-axis. Its period is $\parens*{\frac{2\pi}{\sqrt{3}}, 2\pi}$. It is also reflected over the unit vector $\parens*{\sqrt{3}, 1}$.
\end{itemize}

\end{document}