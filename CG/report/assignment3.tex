\documentclass{article}
\usepackage[utf8]{inputenc}
\usepackage[left=1in,right=1in,top=1in,bottom=1in]{geometry}
\usepackage{crop,graphicx,array,color,flushend,stfloats,amsthm,chngpage,times,fancyhdr,lipsum,lastpage}

%%%%%%%%%%%%   Extra libraries & settings %%%%%%%%%%%%%
\setlength{\parskip}{0.25em}

%%%%%%%%%%%%   Header and Footer  %%%%%%%%%%%%%
\pagestyle{fancy}

\fancypagestyle{plain}{%
  \renewcommand{\headrulewidth}{0pt}%
  \fancyhf{}%
  \fancyfoot[R]{Page \bf\thepage\ \rm of \bf\pageref{LastPage}}%
}


%%%% Customise Titles and Headers: %%%%
\title{Don't forget adding title}
\author{Chinchuthakun Worameth (18B00033)}
\date{\today}

\fancyhf{}
\fancyhead[L]{Chinchuthakun Worameth}
\fancyhead[R]{18B00033}
\fancyfoot[R]{Page \bf\thepage\ \rm of \bf\pageref{LastPage}}


\AtBeginDocument{
%%%%%%%%%%%% Make Title and Format Lines %%%%%%%%%%%%
\maketitle											%
\vspace{-120px}										%
\noindent\rule{\linewidth}{1pt} \par				%
\vspace{100px}										%
\vspace{-20px}										%
\noindent\rule{\linewidth}{1pt} \par				%
\vspace{10px}										%
% %%%%%%%%%%%%%%%%%%% Content %%%%%%%%%%%%%%%%%%%%	
}
\input{math-command}
\usepackage{subfiles}
\usepackage{subcaption}
\usepackage{tikz}
\title{Computer Graphics Assignment \#3}
\begin{document}
\section{Question}
\begin{enumerate}
    \item Explain the aliasing problem caused by discretization with an image prepared by yourself.
    \begin{itemize}
        \item Prepare one image whose size is $512 \times 512$.
        \item Using ``SignalFreq'' program, produce an image that has aliased texture with resampling.
        \item Explain the Nyquist frequency in the case of the resampling.
    \end{itemize}
    \item Explain one anti-aliasing technique to remove the alias on the image produced in the above.
    \begin{itemize}
        \item Design an appropriate filter and describe it by equations in signal and frequency domains.
        \item Explain the procedure of the anti-aliasing, which uses the filter.
        \item Explain the reason why the aliasing is removed by the procedure.
    \end{itemize}
\end{enumerate}
In both answers, include efficient images or graphs to help the understanding of the explanations.

\section{Answer}
\subsection{Aliasing and Nyquist Frequency}
Given the delta function series $\delta_T (x)$, the fourier transform of the discrete sampling function $f_s(x)$ can be derived by Convolution theorem:
\begin{align}
\fourier{f_s(x)}
&= \sint[-\infty][\infty] f(x)\delta_T(x) e^{-i\omega x} \dx \\
&= F(\omega) \ast \fourier{\ssum[n][-\infty][\infty] \delta(t-nT)} \\
&= f^* \ssum[n][-\infty][\infty] F(\omega - nf^*) \label{result}
\end{align}
where $f^* = 1 / T$ denotes the Nyquist frequency. 

From equation \eqref{result}, we can observe that if $F(\omega) > 0$ when $\omega > f^* / 2$, each copy of $F$ will overlap with each other. This situation leads to a phenomenon called \tit{aliasing}, where we cannot distinguish low-frequency components from high-frequency ones due to overlapping in the frequency domain. For example, consider an image with its frequency domain representation as shown in Figure \ref{original}. When we increase sampling interval (or equivalently reduce sampling rate $f^*$), aliasing happens as illustrated in Figure \ref{aliasing}, especially on the roof tiles below the nameplate of the station.

\begin{figure}[h]
    \centering
    \hspace*{\fill}
    \subcaptionbox{An original $512 \times 512$ image with its frequency domain representation \label{original}}{\includegraphics[width=0.45\textwidth]{figures/assignment3/cg3-original.png}}%
    \hspace*{\fill}
    \subcaptionbox{Aliasing happens with a lower Nyquist frequency $f^*$
    \label{aliasing}}{\includegraphics[width=0.45\textwidth]{figures/assignment3/cg3-aliasing.png}}%
    \hspace*{\fill}
    \caption{Relationship between aliasing effect and Nyquist frequency}
\end{figure}

\subsection{Anti-aliasing}
Suppose that the sampling rate is fixed, we can design a low-pass filter to remove the aliasing on the image. For example, we can use a Gaussian filter \eqref{gaussian-signal} and \eqref{gaussian-frequency} to smooth the image, assuming that $\sigma_x = \sigma_y = \sigma$.
\begin{subequations}
    \begin{gather}
        G(x,y) = \frac{1}{2\pi\sigma^2}e^{-\frac{x^2 + y^2}{2\sigma^2}} \label{gaussian-signal} \\
        \fourier{G(x,y)} = e^{-\frac{\sigma^2}{2}\parens*{\omega_x^2 + \omega_y^2}} \label{gaussian-frequency}
    \end{gather}
\end{subequations}

We apply the filter to an image by performing convolution between both of them. Or equivalently, sliding a finite representation of kernel across the image and adding results together in a discretized space. The principle behind this procedure is smoothing and removing high-frequency components from the image. Given that the standard deviation $\sigma$ is chosen appropriately, it is possible to reduce high-frequency components such that $F(\omega) = 0$ when $\omega > f^* / 2$; thus, mitigating the aliasing effect. Figure \ref{gaussian-image} demonstrates the application of the Gaussian filter.

\begin{figure}[h]
    \centering
    \includegraphics[width=0.45\textwidth]{figures/assignment3/cg3-gaussian.png}
    \caption{The smoothed image with less aliasing and its frequency domain representation}
    \label{gaussian-image}
\end{figure}

\end{document}