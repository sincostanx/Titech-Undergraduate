\documentclass{article}
\usepackage[utf8]{inputenc}
\usepackage[left=1in,right=1in,top=1in,bottom=1in]{geometry}
\usepackage{crop,graphicx,array,color,flushend,stfloats,amsthm,chngpage,times,fancyhdr,lipsum,lastpage}

%%%%%%%%%%%%   Extra libraries & settings %%%%%%%%%%%%%
\setlength{\parskip}{0.25em}

%%%%%%%%%%%%   Header and Footer  %%%%%%%%%%%%%
\pagestyle{fancy}

\fancypagestyle{plain}{%
  \renewcommand{\headrulewidth}{0pt}%
  \fancyhf{}%
  \fancyfoot[R]{Page \bf\thepage\ \rm of \bf\pageref{LastPage}}%
}


%%%% Customise Titles and Headers: %%%%
\title{Don't forget adding title}
\author{Chinchuthakun Worameth (18B00033)}
\date{\today}

\fancyhf{}
\fancyhead[L]{Chinchuthakun Worameth}
\fancyhead[R]{18B00033}
\fancyfoot[R]{Page \bf\thepage\ \rm of \bf\pageref{LastPage}}


\AtBeginDocument{
%%%%%%%%%%%% Make Title and Format Lines %%%%%%%%%%%%
\maketitle											%
\vspace{-120px}										%
\noindent\rule{\linewidth}{1pt} \par				%
\vspace{100px}										%
\vspace{-20px}										%
\noindent\rule{\linewidth}{1pt} \par				%
\vspace{10px}										%
% %%%%%%%%%%%%%%%%%%% Content %%%%%%%%%%%%%%%%%%%%	
}
\input{math-command}
\usepackage{subfiles}
\usepackage{subcaption}
\usepackage{tikz}
\title{Computer Graphics Assignment \#2}
\begin{document}
\section{Question}
\begin{enumerate}
    \item Draw the Fourier transform of a 2 Dimensional sampling function, whose intervals along with $x$ and $y$ axes are $\pi$, on a graph as shown in Fig. \ref{frequency-sampling}.
    \item Using the graph, explain the characteristics of the waves, which have the highest frequency in the range of aliasing free.
\end{enumerate}

\section{Answer}

\subsection{Fourier transform of 2-Dimensional Sampling Function}
From the Fourier series of 1-dimensional sampling function, we have
\begin{align}
\ssum[n][-\infty][\infty] \ssum[m][-\infty][\infty] \func[\delta](x-nT,y-mT)
&= \parens*{\ssum[n][-\infty][\infty] \func[\delta](x-nT)} \parens*{\ssum[m][-\infty][\infty] \func[\delta](y-mT)} \\
&= \parens*{\frac{1}{T}\ssum[n][-\infty][\infty] e^{in\omega_{x0}x}} \parens*{\frac{1}{T}\ssum[m][-\infty][\infty] e^{im\omega_{y0}y}} \\
&= \frac{1}{T^2} \ssum[n][-\infty][\infty] \ssum[m][-\infty][\infty] e^{in\omega_{x0}x} e^{im\omega_{y0}y}
\end{align}
Hence, similar to the Fourier transform of 1-dimensional sampling function, we have
\begin{align}
\fourier{\func[\delta][][T](x,y)}
&= \fourier{\ssum[n][-\infty][\infty] \ssum[m][-\infty][\infty] \func[\delta](x-nT,y-mT)} \\
&= \ssum[n][-\infty][\infty] \ssum[m][-\infty][\infty] \sint[-\infty][\infty] \sint[-\infty][\infty] \frac{1}{T^2} e^{in\omega_{x0}x} e^{im\omega_{y0}y} e^{-iw_x x} e^{-iw_y y} \dx \dy \\
&= \frac{1}{T^2} \ssum[n][-\infty][\infty] \ssum[m][-\infty][\infty] e^{-i(\omega_x - n\omega_{x0})x} e^{-i(\omega_y - m\omega_{y0})y} \\
&= \frac{1}{T^2} \ssum[n][-\infty][\infty] \ssum[m][-\infty][\infty] \func[\delta](\omega_x - n\omega_{x0},\omega_y - m\omega_{y0})
\end{align}
Since the intervals along $x$ and $y$ axes are $\pi$, i.e. $\omega_{x0} = \omega={y0} = \pi$, we have
\begin{equation}
\func[\delta][][\pi](\omega_x,\omega_y) = \frac{1}{T^2} \ssum[n][-\infty][\infty] \ssum[m][-\infty][\infty] \func[\delta](\omega_x - n\pi,\omega_y - m\pi) \label{solution}
\end{equation}

We can plot the equation \eqref{solution} in the $xy$-coordinate as shown in Figure \ref{frequency-sampling}. Note that each red point indicate a value of $\frac{1}{T^2}$.
\subfile{./figures/assignment2/graph}

\subsection{Characteristics of Waves}
For a signal $f(x,y)$, we have $\func[f](x,y)\func[\delta][][T](x,y) = \func[F](\omega_x,\omega_y) \ast \func[\delta][][\omega](\omega_x,\omega_y)$. Now, assuming that the shape of $\func[F](\omega_x,\omega_y)$ a bivariate guassian distribution with variance $\sigma_{\omega_x} = f_x$ and $\sigma_{\omega_y} = f_y$ (for the purpose of visualization), the convolution creates many copies of that shape, centered at each blue point, as illustrated in \ref{frequency-sampling-2}. If two circles overlap each other, aliasing will occur, making it impossible to recover the original signal. Therefore, recoverable waves must have the highest frequency in the range of aliasing free; that is, $f_x < \frac{\pi}{2}$ and $f_y < \frac{\pi}{2}$. Note that this intuition concurs with the sampling theorem.

\end{document}